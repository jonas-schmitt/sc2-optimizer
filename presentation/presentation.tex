%%%%%%%%%%%%%%%%%%%%%%%%%%%%%%%%%%%%%%%%%
% Beamer Presentation
% LaTeX Template
% Version 1.0 (10/11/12)
%
% This template has been downloaded from:
% http://www.LaTeXTemplates.com
%
% License:
% CC BY-NC-SA 3.0 (http://creativecommons.org/licenses/by-nc-sa/3.0/)
%
%%%%%%%%%%%%%%%%%%%%%%%%%%%%%%%%%%%%%%%%%

%----------------------------------------------------------------------------------------
%	PACKAGES AND THEMES
%----------------------------------------------------------------------------------------

\documentclass{beamer}

\mode<presentation> {

% The Beamer class comes with a number of default slide themes
% which change the colors and layouts of slides. Below this is a list
% of all the themes, uncomment each in turn to see what they look like.

%\usetheme{default}
%\usetheme{AnnArbor}
%\usetheme{Antibes}
%\usetheme{Bergen}
%\usetheme{Berkeley}
%\usetheme{Berlin}
%\usetheme{Boadilla}
%\usetheme{CambridgeUS}
%\usetheme{Copenhagen}
%\usetheme{Darmstadt}
%\usetheme{Dresden}
%\usetheme{Frankfurt}
%\usetheme{Goettingen}
%\usetheme{Hannover}
%\usetheme{Ilmenau}
%\usetheme{JuanLesPins}
%\usetheme{Luebeck}
%\usetheme{Madrid}
%\usetheme{Malmoe}
%\usetheme{Marburg}
%\usetheme{Montpellier}
%\usetheme{PaloAlto}
%\usetheme{Pittsburgh}
\usetheme{Rochester}
%\usetheme{Singapore}
%\usetheme{Szeged}
%\usetheme{Warsaw}
\usepackage{amsmath}
\usepackage[ngerman]{babel}
\usepackage{lmodern}

% As well as themes, the Beamer class has a number of color themes
% for any slide theme. Uncomment each of these in turn to see how it
% changes the colors of your current slide theme.

%\usecolortheme{albatross}
%\usecolortheme{beaver}
%\usecolortheme{beetle}
%\usecolortheme{crane}
%\usecolortheme{dolphin}
%\usecolortheme{dove}
%\usecolortheme{fly}
%\usecolortheme{lily}
%\usecolortheme{orchid}
%\usecolortheme{rose}
%\usecolortheme{seagull}
%\usecolortheme{seahorse}
%\usecolortheme{whale}
%\usecolortheme{wolverine}

%\setbeamertemplate{footline} % To remove the footer line in all slides uncomment this line
%\setbeamertemplate{footline}[page number] % To replace the footer line in all slides with a simple slide count uncomment this line

%\setbeamertemplate{navigation symbols}{} % To remove the navigation symbols from the bottom of all slides uncomment this line
}

\usepackage{graphicx} % Allows including images
\usepackage{booktabs} % Allows the use of \toprule, \midrule and \bottomrule in tables
\usepackage{epstopdf,epsfig}

%----------------------------------------------------------------------------------------
%	TITLE PAGE
%----------------------------------------------------------------------------------------

\title{A Multi-Objective Genetic Algorithm for Evaluating Build Order Effectiveness in Starcraft II} % The short title appears at the bottom of every slide, the full title is only on the title page

\author{Jonas Schmitt} % Your name
\date{\today} % Date, can be changed to a custom date

%\AtBeginSection[]
%{
% \begin{frame}
% \frametitle{Overview}
% \tableofcontents[currentsection]
% \end{frame}
%}

\begin{document}

\begin{frame}
\titlepage % Print the title page as the first slide
\end{frame}

%\begin{frame}
%\frametitle{Outline} % Table of contents slide, comment this block out to remove it
%\tableofcontents % Throughout your presentation, if you choose to use \section{} and \subsection{} commands, these will automatically be printed on this slide as an overview of your presentation
%\end{frame}
%------------------------------------------------
\begin{frame}{Starcraft II}
\begin{columns}[c] % the "c" option specifies center vertical alignment
    \column{.5\textwidth} % column designated by a command
    \begin{itemize}
		\item Military science-fiction real-time strategy game
		\item \alert{Goal:} Producing the right combination of units (\alert{Macromanagement}) to destroy the other player's units and structures in direct combat (\alert{Micromanagement})
	\end{itemize}
    \column{.5\textwidth}
    	\includegraphics[width=1.0\linewidth]{starcraft-2-screenshot.jpg}
\end{columns}
\begin{itemize}
\item E-Sport scene with growing popularity (price pools up to US\$170,000)
\\ $\Rightarrow$ Importance of \alert{Balancing} (Are all three races equally strong?) 
\end{itemize}
\end{frame}

\begin{frame}{Balancing}
\begin{itemize}
\item \alert{Macromanagement:} Which units can be produced in a certain amount of time?
\\ $\Rightarrow$ `` A Multi-objective Genetic Algorithm for Build Order Optimization in StarCraft II '' by Harald K"ostler and Bj"orn Gmeiner
\item \alert{Micromanagement:} Is it possible to predict which of two groups of units wins in combat?
\\ $\Rightarrow$ No suitable approach for Starcraft II yet
\end{itemize}
\end{frame}

\begin{frame}{Roadmap}
\begin{itemize}
\item \alert{Input:} Build Order, i.e. the list of units that have been produced until a certain point of time in the game
\item \alert{Goal:} Simulate and optimize the behaviour (moving and attacking) of each single unit
\\ $\Rightarrow$ It can be predicted which player would succeed in a combat assuming optimal control
\end{itemize}
\end{frame}

\begin{frame}{Forward Simulation}
    \begin{itemize}
\item No freely available API for controlling units in the game directly
\\ $\Rightarrow$ An efficient forward simulation is required that determines the winner of an encounter based on a finite set of parameters
\item \alert{Idea:} Describe the behaviour of each unit by a number of parametrized \alert{Potential Fields}
\end{itemize}
\begin{center}
  \includegraphics[width=0.45\linewidth]{friend-crop.pdf}
\end{center}
\end{frame}

\begin{frame}{Forward Simulation}
\begin{columns}[c] % the "c" option specifies center vertical alignment
    \column{.5\textwidth} % column designated by a command
  \includegraphics[width=1.0\linewidth]{enemy1-crop.pdf}
    \column{.5\textwidth}
  \includegraphics[width=1.0\linewidth]{enemy2-crop.pdf}
\end{columns}
\end{frame}

\begin{frame}{Optimization}
\begin{itemize}
\item \alert{Goal:} Iteratively optimize the parameters for both opponents' units against each other
\item \alert{Challenges:} 
\begin{itemize}
\item Large search space (at least 16 parameters for each different type of unit) 
\item No knowledge about the relationship between in- and output
\end{itemize}
$\Rightarrow$ \alert{Genetic Algorithms} are suitable search heuristics for problems of this type
\end{itemize}
\end{frame}

\begin{frame}{Optimization}
\begin{itemize}
\item Binary encode the parameters of both opponents
\item Choose suitable starting values as optimization objective (strategy used by the opponent) for both populations
\item Replace the objective every $n$ generation by the respective optimum and reevaluate the whole population
\item The obtained optima can be used to evaluate the effectiveness of both build orders against the respective other one
\end{itemize}
\end{frame}

\begin{frame}{Optimization}
\begin{enumerate}
\item \alert{Single-Objective Genetic Algorithm:}
The fitness of each individual is approximated with a simple formula: 
\begin{equation*}
		\text{Fitness}(x) =  \frac{\sum\limits_{i=1}^{n} \text{damage}(i) + \sum\limits_{j=1}^{m} \text{health}(j)}{2}
\end{equation*}
$\Rightarrow$ Determine the best combination of genetic operators
\item \alert{Multi-Objective Genetic Algorithm:} Considers all relevant objectives (applied damage, remaining health, value of units killed, value of units remaining etc.)
\\ $\Rightarrow$ Perform the actual optimization using NSGA-II
\end{enumerate}

\end{frame}




%\begin{frame}{Task scheduling in concurrent computer environments}
%Optimization goals:
%\begin{itemize}
%	\item[1)] Equally distribute the workload over the available processors 
%	\item[2)] Minimize the communication between them 
%\end{itemize}
%\vspace*{0.5cm}
%Can be formulated as Graph Partitioning problem:
%%Graph partitioning for use in concurrent computer environments
%\begin{itemize}
%	\item[1)] \textcolor{red}{Vertices} represent \textcolor{red}{computational} costs \\
%	\item[2)] \textcolor{blue}{Edges} represent \textcolor{blue}{communication} costs \\
%\end{itemize}
%\end{frame}




%\section{References}
%\begin{frame}
%\frametitle{References}
%\footnotesize{
%\begin{thebibliography}{99} % Beamer does not support BibTeX so references must be inserted manually as below
%
%\input{biblio}
%
%%\bibitem[Smith, 2012]{p1} John Smith (2012)
%%\newblock Title of the publication
%%\newblock \emph{Journal Name} 12(3), 45 -- 678.
%\end{thebibliography}
%}
%\end{frame}

%------------------------------------------------

\begin{frame}
\Huge{\centerline{The End}}
\end{frame}

%----------------------------------------------------------------------------------------

\end{document}
